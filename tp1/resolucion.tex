\documentclass[12pt]{article}
\usepackage[utf8]{inputenc}
\usepackage[spanish]{babel}
\usepackage{amsmath}
\usepackage{geometry}
\geometry{margin=2.5cm}
\usepackage{fancyhdr}
\usepackage{graphicx}
\usepackage{listings}
\usepackage{float}
\usepackage{multicol}
\usepackage{enumitem}
\usepackage{hyperref}
\usepackage{booktabs}
\usepackage{amssymb}
\usepackage[table]{xcolor}
\usepackage{pifont}
\usepackage{amssymb}
\newcommand{\xmark}{\ding{55}}


\title{Resolución del Trabajo Práctico Nº 1\\\large Sistemas de Numeración}
\date{\date{Abril 2025}} % O reemplazalo con la fecha de entrega exacta

\begin{document}
	
\maketitle

\section*{Nota}
El enunciado original de este trabajo práctico se encuentra en el archivo \href{./enunciado.pdf}{enunciado.pdf} incluido en este repositorio.

\vspace{0.5cm}

A continuación, se desarrollan los ejercicios correspondientes.

\section*{Ejercicio 1: Sistemas numéricos}

% Aquí escribís tu resolución
Indicar a qué sistemas numéricos (binario, octal, decimal, hexadecimal) pueden pertenecer los siguientes números...

\begin{center}
    \begin{tabular}{ccccccccc}
        1011 & 7806 & 9B4 & 85A2 & 1230 & 567 & FFF & ABCDE & 999 \\
    \end{tabular}
\end{center}

\textbf{Resolución:} \\

\begin{table}[H]
	\centering
	\renewcommand{\arraystretch}{1.2}
	\begin{tabular}{cccccc}
		\toprule
		\textbf{Número} & \textbf{Binario} & \textbf{Octal} & \textbf{Decimal} & \textbf{Hexadecimal} & \textbf{Siguiente(s)} \\
		\midrule
			1011    & \checkmark & \checkmark & \checkmark & \checkmark & Bin: 1100, Oct: 1012, Dec: 1012, Hex: 1012 \\
			7806    & \xmark     & \xmark     & \checkmark & \checkmark & Dec: 7807, Hex: 7807 \\
			9B4     & \xmark     & \xmark     & \xmark     & \checkmark & Hex: 9B5 \\
			85A2    & \xmark     & \xmark     & \xmark     & \checkmark & Hex: 85A3 \\
			1230    & \xmark     & \checkmark & \checkmark & \checkmark & Oct: 1231, Dec: 1231, Hex: 1231 \\
			567     & \xmark     & \checkmark & \checkmark & \checkmark & Oct: 570, Dec: 568, Hex: 568 \\
			FFF     & \xmark     & \xmark     & \xmark     & \checkmark & Hex: 1000 \\
			ABCDE   & \xmark     & \xmark     & \xmark     & \checkmark & Hex: ABDCF \\
			999     & \xmark     & \xmark     & \checkmark & \checkmark & Dec: 1000, Hex: 99A \\
	\bottomrule
	\end{tabular}
	\caption{Verificación de representación numérica en diferentes sistemas}
\end{table}


\section*{Ejercicio 2: Interpretación del número \texttt{10}}

Dado el número \texttt{10}, convertirlo a base 10 suponiendo que el mismo está:

\begin{enumerate}[label=\alph*)]
    \item En base 2
    \item En base 5
    \item En base 8
    \item En base 16
\end{enumerate}

\textbf{Resolución:} \\


\begin{table}[H]
	\centering
	\renewcommand{\arraystretch}{1.3}
	\begin{tabular}{|c|c|c|}
		\hline
		\textbf{Interpretado como} & \textbf{Forma de resolución} & \textbf{Resultado en base 10} \\
		\hline
		$10_2$  & $1 \times 2^1 + 0 \times 2^0 = 2 + 0$     & 2  \\
		\hline
		$10_5$  & $1 \times 5^1 + 0 \times 5^0 = 5 + 0$     & 5  \\
		\hline
		$10_8$  & $1 \times 8^1 + 0 \times 8^0 = 8 + 0$     & 8  \\
		\hline
		$10_{16}$ & $1 \times 16^1 + 0 \times 16^0 = 16 + 0$ & 16 \\
		\hline
	\end{tabular}
	\caption{Interpretación del número \texttt{10} en diferentes sistemas de numeración}
\end{table}
	


\end{document}
