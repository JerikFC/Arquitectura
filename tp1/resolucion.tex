\documentclass[12pt]{article}
\usepackage[utf8]{inputenc}
\usepackage[spanish]{babel}
\usepackage{amsmath}
\usepackage{geometry}
\geometry{margin=2.5cm}
\usepackage{fancyhdr}
\usepackage{graphicx}
\usepackage{listings}
\usepackage{float}
\usepackage{multicol}
\usepackage{enumitem}
\usepackage{hyperref}
\usepackage{booktabs}
\usepackage{amssymb}
\usepackage[table]{xcolor}
\usepackage{pifont}
\usepackage{amssymb}
\newcommand{\xmark}{\ding{55}}


\title{Resolución del Trabajo Práctico Nº 1\\\large Sistemas de Numeración}
\date{\date{Abril 2025}} % O reemplazalo con la fecha de entrega exacta

\begin{document}
	
\maketitle

\section*{Nota}
El enunciado original de este trabajo práctico se encuentra en el archivo \href{./enunciado.pdf}{enunciado.pdf} incluido en este repositorio.

\vspace{0.5cm}

A continuación, se desarrollan los ejercicios correspondientes.

\section*{Ejercicio 1: Sistemas numéricos}


Indicar a qué sistemas numéricos (binario, octal, decimal, hexadecimal) pueden pertenecer los siguientes números...

\begin{center}
    \begin{tabular}{ccccccccc}
        1011 & 7806 & 9B4 & 85A2 & 1230 & 567 & FFF & ABCDE & 999 \\
    \end{tabular}
\end{center}

\textbf{Resolución:} \\

\begin{table}[H]
	\centering
	\renewcommand{\arraystretch}{1.2}
	\begin{tabular}{cccccc}
		\toprule
		\textbf{Número} & \textbf{Binario} & \textbf{Octal} & \textbf{Decimal} & \textbf{Hexadecimal} & \textbf{Siguiente(s)} \\
		\midrule
			1011    & \checkmark & \checkmark & \checkmark & \checkmark & Bin: 1100, Oct: 1012, Dec: 1012, Hex: 1012 \\
			7806    & \xmark     & \xmark     & \checkmark & \checkmark & Dec: 7807, Hex: 7807 \\
			9B4     & \xmark     & \xmark     & \xmark     & \checkmark & Hex: 9B5 \\
			85A2    & \xmark     & \xmark     & \xmark     & \checkmark & Hex: 85A3 \\
			1230    & \xmark     & \checkmark & \checkmark & \checkmark & Oct: 1231, Dec: 1231, Hex: 1231 \\
			567     & \xmark     & \checkmark & \checkmark & \checkmark & Oct: 570, Dec: 568, Hex: 568 \\
			FFF     & \xmark     & \xmark     & \xmark     & \checkmark & Hex: 1000 \\
			ABCDE   & \xmark     & \xmark     & \xmark     & \checkmark & Hex: ABDCF \\
			999     & \xmark     & \xmark     & \checkmark & \checkmark & Dec: 1000, Hex: 99A \\
	\bottomrule
	\end{tabular}
	\caption{Verificación de representación numérica en diferentes sistemas}
\end{table}


\section*{Ejercicio 2: Interpretación del número \texttt{10}}

Dado el número \texttt{10}, convertirlo a base 10 suponiendo que el mismo está:

\begin{enumerate}[label=\alph*)]
    \item En base 2
    \item En base 5
    \item En base 8
    \item En base 16
\end{enumerate}

\textbf{Resolución:} \\


\begin{table}[H]
	\centering
	\renewcommand{\arraystretch}{1.3}
	\begin{tabular}{|c|c|c|}
		\hline
		\textbf{Interpretado como} & \textbf{Forma de resolución} & \textbf{Resultado en base 10} \\
		\hline
		$10_2$  & $1 \times 2^1 + 0 \times 2^0 = 2 + 0$     & 2  \\
		\hline
		$10_5$  & $1 \times 5^1 + 0 \times 5^0 = 5 + 0$     & 5  \\
		\hline
		$10_8$  & $1 \times 8^1 + 0 \times 8^0 = 8 + 0$     & 8  \\
		\hline
		$10_{16}$ & $1 \times 16^1 + 0 \times 16^0 = 16 + 0$ & 16 \\
		\hline
	\end{tabular}
	\caption{Interpretación del número \texttt{10} en diferentes sistemas de numeración}
\end{table}

\section*{3) Transformar los siguientes números decimales en:}

	\begin{itemize}
		\item[a)] Números binarios
		\item[b)] Números octales
		\item[c)] Números hexadecimales
	\end{itemize}

	\vspace{0.5cm}

	\noindent
	384 \hspace{1cm} 1259 \hspace{1cm} 111 \hspace{1cm} 0{,}175 \hspace{1cm} 1024 \hspace{1cm} 16 \hspace{1cm} 37{,}25

	\vspace{0.5cm}

	\noindent
	Realizarlo por el método de las divisiones.\\\\




\textbf{Resolución:} \\

\subsection*{a) Conversión de números decimales a binario}
\begin{table}[H]
\centering
\begin{minipage}[t]{0.32\textwidth}
\centering
\begin{tabular}{lll}
\toprule
\textbf{División} & \textbf{Coc.} & \textbf{Res.} \\
\midrule
384 ÷ 2 & 192 & 0 \\
192 ÷ 2 & 96  & 0 \\
96 ÷ 2  & 48  & 0 \\
48 ÷ 2  & 24  & 0 \\
24 ÷ 2  & 12  & 0 \\
12 ÷ 2  & 6   & 0 \\
6 ÷ 2   & 3   & 0 \\
3 ÷ 2   & 1   & 1 \\
1 ÷ 2   & 0   & 1 \\
\bottomrule
\end{tabular}
\end{minipage}
\hfill
\begin{minipage}[t]{0.32\textwidth}
\centering
\begin{tabular}{lll}
\toprule
\textbf{División} & \textbf{Coc.} & \textbf{Res.} \\
\midrule
1259 ÷ 2 & 629 & 1 \\
629 ÷ 2  & 314 & 1 \\
314 ÷ 2  & 157 & 0 \\
157 ÷ 2  & 78  & 1 \\
78 ÷ 2   & 39  & 0 \\
39 ÷ 2   & 19  & 1 \\
19 ÷ 2   & 9   & 1 \\
9 ÷ 2    & 4   & 1 \\
4 ÷ 2    & 2   & 0 \\
2 ÷ 2    & 1   & 0 \\
1 ÷ 2    & 0   & 1 \\
\bottomrule
\end{tabular}
\end{minipage}
\hfill
\begin{minipage}[t]{0.32\textwidth}
\centering
\begin{tabular}{lll}
\toprule
\textbf{División} & \textbf{Coc.} & \textbf{Res.} \\
\midrule
111 ÷ 2 & 55 & 1 \\
55 ÷ 2  & 27 & 1 \\
27 ÷ 2  & 13 & 1 \\
13 ÷ 2  & 6  & 1 \\
6 ÷ 2   & 3  & 0 \\
3 ÷ 2   & 1  & 1 \\
1 ÷ 2   & 0  & 1 \\
\bottomrule
\end{tabular}
\end{minipage}
\end{table}


\begin{table}[H]
\centering
\begin{minipage}[t]{0.32\textwidth}
\centering
\begin{tabular}{lll}
\toprule
\textbf{División} & \textbf{Coc.} & \textbf{Res.} \\
\midrule
1024 ÷ 2 & 512 & 0 \\
512 ÷ 2  & 256 & 0 \\
256 ÷ 2  & 128 & 0 \\
128 ÷ 2  & 64  & 0 \\
64 ÷ 2   & 32  & 0 \\
32 ÷ 2   & 16  & 0 \\
16 ÷ 2   & 8   & 0 \\
8 ÷ 2    & 4   & 0 \\
4 ÷ 2    & 2   & 0 \\
2 ÷ 2    & 1   & 0 \\
1 ÷ 2    & 0   & 1 \\
\bottomrule
\end{tabular}
\end{minipage}
\hfill
\begin{minipage}[t]{0.32\textwidth}
\centering
\begin{tabular}{lll}
\toprule
\textbf{División} & \textbf{Coc.} & \textbf{Res.} \\
\midrule
16 ÷ 2 & 8 & 0 \\
8 ÷ 2  & 4 & 0 \\
4 ÷ 2  & 2 & 0 \\
2 ÷ 2  & 1 & 0 \\
1 ÷ 2  & 0 & 1 \\
\bottomrule
\end{tabular}
\end{minipage}
\hfill
\begin{minipage}[t]{0.32\textwidth}
\centering
\begin{tabular}{lll}
\toprule
\textbf{División} & \textbf{Coc.} & \textbf{Res.} \\
\midrule
37 ÷ 2 & 18 & 1 \\
18 ÷ 2 & 9  & 0 \\
9 ÷ 2  & 4  & 1 \\
4 ÷ 2  & 2  & 0 \\
2 ÷ 2  & 1  & 0 \\
1 ÷ 2  & 0  & 1 \\
\bottomrule
\end{tabular}
\end{minipage}
\end{table}



\begin{table}[H]
\centering
\begin{minipage}[t]{0.48\textwidth}
\centering
\begin{tabular}{lll}
\toprule
\textbf{Número × 2} & \textbf{Part E.} & \textbf{Nueva F.} \\
\midrule
0.175 × 2 = 0.35 & 0 & 0.35 \\
0.35 × 2 = 0.70 & 0 & 0.70 \\
0.70 × 2 = 1.40 & 1 & 0.40 \\
0.40 × 2 = 0.80 & 0 & 0.80 \\
0.80 × 2 = 1.60 & 1 & 0.60 \\
0.60 × 2 = 1.20 & 1 & 0.20 \\
0.20 × 2 = 0.40 & 0 & 0.40 \\
0.40 × 2 = 0.80 & 0 & 0.80 \\


\bottomrule
\end{tabular}
\end{minipage}
\hfill
\begin{minipage}[t]{0.48\textwidth}
\centering
\begin{tabular}{lll}
\midrule
\textbf{Número × 2} & \textbf{Part E.} & \textbf{Nueva F.} \\
\midrule
0.25 × 2 = 0.50 & 0 & 0.50 \\
0.50 × 2 = 1.00 & 1 & 0.00 \\
\bottomrule
\end{tabular}
\end{minipage}
\end{table}





\subsection*{b) Conversión de números decimales a octales}


\begin{table}[H]
\centering
\begin{minipage}[t]{0.32\textwidth}
\centering
\begin{tabular}{lll}
\toprule
\textbf{División} & \textbf{Coc.} & \textbf{Res.} \\
\midrule
384 ÷ 8 & 48 & 0 \\
48 ÷ 8 & 6 & 0 \\
6 ÷ 8 & 0 & 6 \\
\bottomrule
\end{tabular}
\end{minipage}
\hfill
\begin{minipage}[t]{0.32\textwidth}
\centering
\begin{tabular}{lll}
\toprule
\textbf{División} & \textbf{Coc.} & \textbf{Res.} \\
\midrule
1259 ÷ 8 & 157 & 3 \\
157 ÷ 8 & 19 & 5 \\
19 ÷ 8 & 2 & 3 \\
2 ÷ 8 & 0 & 2 \\
\bottomrule
\end{tabular}
\end{minipage}
\hfill
\begin{minipage}[t]{0.32\textwidth}
\centering
\begin{tabular}{lll}
\toprule
\textbf{División} & \textbf{Coc.} & \textbf{Res.} \\
\midrule
111 ÷ 8 & 13 & 7 \\
13 ÷ 8 & 1 & 5 \\
1 ÷ 8 & 0 & 1 \\
\bottomrule
\end{tabular}
\end{minipage}
\end{table}


\begin{table}[H]
\centering
\begin{minipage}[t]{0.32\textwidth}
\centering
\begin{tabular}{lll}
\toprule
\textbf{División} & \textbf{Coc.} & \textbf{Res.} \\
\midrule
1024 ÷ 8 & 128 & 0 \\
128 ÷ 8 & 16 & 0 \\
16 ÷ 8 & 2 & 0 \\
2 ÷ 8 & 0 & 2 \\
\bottomrule
\end{tabular}
\end{minipage}
\hfill
\begin{minipage}[t]{0.32\textwidth}
\centering
\begin{tabular}{lll}
\toprule
\textbf{División} & \textbf{Coc.} & \textbf{Res.} \\
\midrule
16 ÷ 8 & 2 & 0 \\
2 ÷ 8 & 0 & 2 \\
\bottomrule
\end{tabular}
\end{minipage}
\hfill
\begin{minipage}[t]{0.32\textwidth}
\centering
\begin{tabular}{lll}
\toprule
\textbf{División} & \textbf{Coc.} & \textbf{Res.} \\
\midrule
37 ÷ 8 & 4 & 5 \\
4 ÷ 8 & 0 & 4 \\
\bottomrule
\end{tabular}
\end{minipage}
\end{table}
	

\begin{table}[H]
\centering
\begin{minipage}[t]{0.48\textwidth}
\centering
\begin{tabular}{lll}
\toprule
\textbf{Número × 8} & \textbf{Part E.} & \textbf{Nueva F.} \\
\midrule
0.175 × 8 = 1.4 & 1 & 0.4 \\
0.4 × 8 = 3.2 & 3 & 0.2 \\
0.2 × 8 = 1.6 & 1 & 0.6 \\
0.6 × 8 = 4.8 & 4 & 0.8 \\
0.8 × 8 = 6.4 & 6 & 0.4 \\
0.4 × 8 = 3.2 & 3 & 0.2 \\
\bottomrule
\end{tabular}
\end{minipage}
\hfill
\begin{minipage}[t]{0.48\textwidth}
\centering
\begin{tabular}{lll}
\midrule
\textbf{Número × 8} & \textbf{Part E.} & \textbf{Nueva F.} \\
\midrule
0.25 × 8 = 2.00 & 2 & 0.00 \\
\bottomrule
\end{tabular}
\end{minipage}
\end{table}



\subsection*{c) Conversión de números decimales a hexadecimales}


\begin{table}[H]
\centering
\begin{minipage}[t]{0.32\textwidth}
\centering
\begin{tabular}{lll}
\toprule
\textbf{División} & \textbf{Coc.} & \textbf{Res.} \\
\midrule
384 ÷ 16 & 24 & 0 \\
24 ÷ 16 & 1 & 8 \\
1 ÷ 16 & 0 & 1 \\
\bottomrule
\end{tabular}
\end{minipage}
\hfill
\begin{minipage}[t]{0.32\textwidth}
\centering
\begin{tabular}{lll}
\toprule
\textbf{División} & \textbf{Coc.} & \textbf{Res.} \\
\midrule
1259 ÷ 16 & 78 & B \\
78 ÷ 16 & 4 & E \\
4 ÷ 16 & 0 & 4 \\
\bottomrule
\end{tabular}
\end{minipage}
\hfill
\begin{minipage}[t]{0.32\textwidth}
\centering
\begin{tabular}{lll}
\toprule
\textbf{División} & \textbf{Coc.} & \textbf{Res.} \\
\midrule
111 ÷ 16 & 6 & F \\
6 ÷ 16 & 0 & 6 \\
\bottomrule
\end{tabular}
\end{minipage}
\end{table}


\begin{table}[H]
\centering
\begin{minipage}[t]{0.32\textwidth}
\centering
\begin{tabular}{lll}
\toprule
\textbf{División} & \textbf{Coc.} & \textbf{Res.} \\
\midrule
1024 ÷ 16 & 64 & 0 \\
64 ÷ 16 & 4 & 0 \\
4 ÷ 16 & 0 & 4 \\
\bottomrule
\end{tabular}
\end{minipage}
\hfill
\begin{minipage}[t]{0.32\textwidth}
\centering
\begin{tabular}{lll}
\toprule
\textbf{División} & \textbf{Coc.} & \textbf{Res.} \\
\midrule
16 ÷ 16 & 1 & 0 \\
1 ÷ 16 & 0 & 1 \\
\bottomrule
\end{tabular}
\end{minipage}
\hfill
\begin{minipage}[t]{0.32\textwidth}
\centering
\begin{tabular}{lll}
\toprule
\textbf{División} & \textbf{Coc.} & \textbf{Res.} \\
\midrule
37 ÷ 16 & 2 & 5 \\
2 ÷ 16 & 0 & 2 \\
\bottomrule
\end{tabular}
\end{minipage}
\end{table}
	

\begin{table}[H]
\centering
\begin{minipage}[t]{0.48\textwidth}
\centering
\begin{tabular}{lll}
\toprule
\textbf{Número × 8} & \textbf{Part E.} & \textbf{Nueva F.} \\
\midrule
0.175 × 16 = 2.8 & 2 & 0.8 \\
0.8 × 16 = 12.8 & C & 0.8 \\
0.8 × 16 = 12.8 & C & 0.8 \\
\bottomrule
\end{tabular}
\end{minipage}
\hfill
\begin{minipage}[t]{0.48\textwidth}
\centering
\begin{tabular}{lll}
\midrule
\textbf{Número × 8} & \textbf{Part E.} & \textbf{Nueva F.} \\
\midrule
0.25 × 16 = 4.00 & 4 & 0.00 \\
\bottomrule
\end{tabular}
\end{minipage}
\end{table}


\subsection*{Resultado final}

\begin{center}
\begin{tabular}{|c|c|c|c|}
\hline
\textbf{Decimal} & \textbf{Binario} & \textbf{Octal} & \textbf{Hexadecimal} \\
\hline
384 & 110000000 & 600 & 180 \\
1259 & 10011101011 & 2353 & 4EB \\
111 & 1101111 & 157 & 6F \\
1024 & 10000000000 & 2000 & 400 \\
16 & 10000 & 20 & 10 \\
37.25 & 100101.01 & 45.2 & 25.4 \\
0.175 & \(0.001011\overline{0011}\) & \(0.1\overline{3146}\) & \(0.2\overline{C}\) \\
\hline
\end{tabular}
\end{center}

\section*{4) Pasar al sistema decimal los siguientes números:}
\begin{center}
110111$_b$\hspace{1cm}3AF$_h$\hspace{1cm}223{,}274$_o$\hspace{1cm}F0F0{,}EA$_h$\hspace{1cm}2$_o$\\
2$_h$\hspace{1cm}111101{,}101101$_b$\hspace{1cm}101F{,}25$_h$
\end{center}

Realizarlo por descomposición en el polinomio equivalente.\\\\\\

\textbf{Resolución:} \\

\begin{table}[H]
	\centering
	\renewcommand{\arraystretch}{1.3}
	\begin{tabular}{|c|c|c|}
		\hline
		\textbf{Interpretado como} & \textbf{Forma de resolución} & \textbf{Resultado en base 10} \\
		\hline
		$110111_b$ & $1 \times 2^5 + 1 \times 2^4 + 0 \times 2^3 + 1 \times 2^2 + 1 \times 2^1 + 1 \times 2^0$ & 55 \\
		\hline
		$3AF_h$ & $3 \times 16^2 + 10 \times 16^1 + 15 \times 16^0$ & 943 \\
		\hline
		$223.274_o$ & $2 \times 8^2 + 2 \times 8^1 + 3 + 2 \times 8^{-1} + 7 \times 8^{-2} + 4 \times 8^{-3}$ & 147.3671875 \\
		\hline
		$F0F0.EA_h$ & $61440 + 240 + 0.875 + 0.0390625$ & 61680.9140625 \\
		\hline
		$2_o$ & $2$ & 2 \\
		\hline
		$2_h$ & $2$ & 2 \\
		\hline
		$111101.101101_b$ & $61 + 0.703125$ & 61.703125 \\
		\hline
		$101F.25_h$ & $4127 + 0.14453125$ & 4127.14453125 \\
		\hline
	\end{tabular}
	\caption{Conversión de varios números en distintas bases a base decimal}
\end{table}



\end{document}
