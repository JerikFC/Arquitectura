\documentclass[12pt]{article}
\usepackage[utf8]{inputenc}
\usepackage[spanish]{babel}
\usepackage{amsmath}
\usepackage{geometry}
\geometry{margin=2.5cm}
\usepackage{fancyhdr}
\usepackage{graphicx}
\usepackage{listings}
\usepackage{float}
\usepackage{multicol}
\usepackage{enumitem}
\usepackage{hyperref}
\usepackage{amssymb}
\usepackage{makecell}



\title{Resolución del Trabajo Práctico Nº 2\\\large Representación de Datos (Primera Parte)}
\date{\date{Mayo 2025}}
\begin{document}
\maketitle
\section*{Nota}

El enunciado original de este trabajo práctico se encuentra en el archivo \href{./enunciado.pdf}{enunciado.pdf} incluido en este repositorio.

\vspace{0.5cm}

A continuación, se desarrollan los ejercicios correspondientes.


\section{Resolución de sumas en octal}

\[
\begin{array}{r}
\text{a)}
  1527 \\
+ \ 183 \\
\hline
  1662
\end{array}
\quad\
\begin{array}{r}
\text{b)}
  17406 \\
+ \ 63065 \\
\hline
  102462
\end{array}
\quad\
\begin{array}{r}
\text{c)}
  365 \\
+ 23 \\
\hline
  410
\end{array}
\quad\
\begin{array}{r}
    \text{d)}
  2732 \\
+ 1265 \\
\hline
  4217
\end{array}
\]

\section{Resolución de sumas en Hexadecimal}

\[
\begin{array}{r}
\text{a)}
  B359 \\
+ \ 83A \\
\hline
  BB93
\end{array}
\quad\
\begin{array}{r}
\text{b)}
  AB350 \\
+ \ 0123 \\
\hline
  AB473
\end{array}
\quad\
\begin{array}{r}
\text{c)}
  AF \\
+ C3 \\
\hline
  172
\end{array}
\quad\
\begin{array}{r}
    \text{d)}
  174 \\
+ 3C \\
\hline
  1B0
\end{array}
\quad\
\begin{array}{r}
    \text{e)}
    20F5 \\
    + 31B\\
    \hline
    2410
\end{array}
\quad\
\begin{array}{r}
    \text{f)}
    2E70 \\
    + AA7F \\
    \hline
    D8EF      
\end{array}
\]

\section{Resolución de A + B, con A = 110 y B = 1101 en distintos sistemas}

\[
\begin{array}{r}
\text{a)}
  1110 \\
+ \ 1101 \\
\hline
  11011
\end{array}
\quad\
\begin{array}{r}
\text{b)}
  1110 \\
+ \ 1101 \\
\hline
  2211
\end{array}
\quad\
\begin{array}{r}
\text{c)}
  1110 \\
+ 1101 \\
\hline
  2211
\end{array}
\quad\
\begin{array}{r}
    \text{d)}
  1110 \\
+ 1101 \\
\hline
  2211
\end{array}
\]

\vspace{2cm}

\section{Resolución de representar números enteros no signados con formato 8 bits}
\begin{table}[h!]
  \centering
  \begin{tabular}{|c|c|}  % El | crea líneas verticales entre las columnas
  \hline
  \textbf{Número Decimal} & \textbf{Representación Binaria (8 bits)} \\
  \hline  % Línea horizontal después del encabezado
  247 & 1111\ 0111 \\
  \hline
  33  & 0010\ 0001 \\
  \hline
  279 & \makecell[l]{No se puede representar en formato de 8 bits\\ sin signo porque excede el límite máximo de Representación\\ de ese formato que va de 0 - 255} \\
  \hline
  128 & 1000\ 0000 \\
  \hline
  111 & 0110\ 1111 \\
  \hline  % Línea horizontal al final de la tabla
  \end{tabular}
  \caption{Tabla de Representación Binaria}
  \end{table}
    

 \section{Resolución de expresar los números signados en distintos convenios con formato 8 bits}
\begin{table}[h!]
  \centering
  \begin{tabular}{|c|c|c|c|c|}  % Tabla con 5 columnas y líneas verticales
  \hline
  \textbf{Núm} & \makecell[c]{Número Positivo\\(En caso se pueda)} & \textbf{SyM} & \textbf{C1} & \textbf{C2 } \\
  \hline
  -35 & 00100011 & 10100011 & 11011100 & 11011101 \\\hline
  122 & 01111010 & 01111010 & 10000101 & 10000110 \\\hline
  -136 & \multicolumn{4}{|c|}{\makecell[l]{No se puede representar en ningún formato de los 3,\\porque -136 está fuera del rango permitido para el formato 8 bits}} \\  % Fila 4 con contenido corrido
  \hline
  -55 & 00110111 & 10110111 & 11001000 & 11001001 \\\hline
  -128 & No es posible, excede el rango & No es posible & No es posible & 10000000 \\
  \hline
  -127 & 01111111 & 11111111 & 10000000 & 10000001 \\\hline
  128 & No es posible, excede el rango & No es posible & No es posible & 10000000 \\
  \hline
  \end{tabular}
  \caption{Tabla con los distintos convenios}
\end{table}


\vspace{4cm}
\section{Resolución de qué número decimal es si están considerados en distintos convenios}

\begin{table}[h!]
  \centering
  \begin{tabular}{|c|c|c|c|c|}
  \hline
  \textbf{Núm} & \makecell[c]{No Signado} & \textbf{Signado SyM} & \textbf{Signado C1} & \textbf{Signado C2 } \\
  \hline
  11001100 & \makecell[c]{11001100\\ 204} & \makecell[c]{11001100\\-76} & \makecell[c]{00110011\\-51}  & \makecell[c]{00110100\\-52} \\\hline
  10101010 & \makecell[c]{10101010\\ 170} & \makecell[c]{10101010\\-42} & \makecell[c]{01010101\\-85}  & \makecell[c]{01010110\\-86} \\\hline
  01111001 & \makecell[c]{01111001\\ 21} & \makecell[c]{01111001\\21} & \makecell[c]{01111001\\21}  & \makecell[c]{01111001\\21} \\\hline
  1111111111001100 & \makecell[c]{1111111111001100\\65484 } & \makecell[c]{1111111111001100\\-32652} & \makecell[c]{0000000000110011\\-51}  & \makecell[c]{0000000000110100\\-52} \\\hline
  \end{tabular}
  \caption{Tabla con los distintos convenios}
\end{table}



























\end{document}