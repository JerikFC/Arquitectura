\documentclass[12pt]{article}
\usepackage[utf8]{inputenc}
\usepackage[spanish]{babel}
\usepackage{amsmath}
\usepackage{geometry}
\geometry{margin=2.5cm}
\usepackage{fancyhdr}
\usepackage{graphicx}
\usepackage{listings}
\usepackage{float}
\usepackage{multicol}
\usepackage{enumitem}
\usepackage{hyperref}


\title{Resolución del Trabajo Práctico Nº 2\\\large Representación de Datos (Primera Parte)}
\date{\date{Mayo 2025}}
\begin{document}
\maketitle
\section*{Nota}

El enunciado original de este trabajo práctico se encuentra en el archivo \href{./enunciado.pdf}{enunciado.pdf} incluido en este repositorio.

\vspace{0.5cm}

A continuación, se desarrollan los ejercicios correspondientes.


\section{Resolución de sumas en octal}

\[
\begin{array}{r}
\text{a)}
  1527 \\
+ \ 183 \\
\hline
  1662
\end{array}
\quad\
\begin{array}{r}
\text{b)}
  17406 \\
+ \ 63065 \\
\hline
  102462
\end{array}
\quad\
\begin{array}{r}
\text{c)}
  365 \\
+ 23 \\
\hline
  410
\end{array}
\quad\
\begin{array}{r}
    \text{d)}
  2732 \\
+ 1265 \\
\hline
  4217
\end{array}
\]

\section{Resolución de sumas en Hexadecimal}

\[
\begin{array}{r}
\text{a)}
  B359 \\
+ \ 83A \\
\hline
  BB93
\end{array}
\quad\
\begin{array}{r}
\text{b)}
  AB350 \\
+ \ 0123 \\
\hline
  AB473
\end{array}
\quad\
\begin{array}{r}
\text{c)}
  AF \\
+ C3 \\
\hline
  172
\end{array}
\quad\
\begin{array}{r}
    \text{d)}
  174 \\
+ 3C \\
\hline
  1B0
\end{array}
\quad\
\begin{array}{r}
    \text{e)}
    20F5 \\
    + 31B\\
    \hline
    2410
\end{array}
\quad\
\begin{array}{r}
    \text{f)}
    2E70 \\
    + AA7F \\
    \hline
    D8EF      
\end{array}
\]

\section{Resolución de A + B, con A = 110 y B = 1101 en distintos sistemas}

\[
\begin{array}{r}
\text{a)}
  1110 \\
+ \ 1101 \\
\hline
  11011
\end{array}
\quad\
\begin{array}{r}
\text{b)}
  1110 \\
+ \ 1101 \\
\hline
  2211
\end{array}
\quad\
\begin{array}{r}
\text{c)}
  1110 \\
+ 1101 \\
\hline
  2211
\end{array}
\quad\
\begin{array}{r}
    \text{d)}
  1110 \\
+ 1101 \\
\hline
  2211
\end{array}
\]

\end{document}